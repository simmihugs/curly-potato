% Created 2024-05-20 Mon 23:26
% Intended LaTeX compiler: pdflatex
\documentclass[11pt]{article}
\usepackage[utf8]{inputenc}
\usepackage[T1]{fontenc}
\usepackage{graphicx}
\usepackage{longtable}
\usepackage{wrapfig}
\usepackage{rotating}
\usepackage[normalem]{ulem}
\usepackage{amsmath}
\usepackage{amssymb}
\usepackage{capt-of}
\usepackage{hyperref}
\usepackage[a4paper, margin=0.5in, top=.5in, bottom=.5in]{geometry}
\usepackage{titletoc}
\usepackage{wrapfig}
\usepackage[export]{adjustbox}
\author{Simon Grätz}
\date{\today}
\title{Worksheet 3}
\hypersetup{
 pdfauthor={Simon Grätz},
 pdftitle={Worksheet 3},
 pdfkeywords={},
 pdfsubject={},
 pdfcreator={Emacs 27.1 (Org mode 9.7-pre)}, 
 pdflang={English}}
\begin{document}

\maketitle
\tableofcontents

\section{Peterson Lock}
\label{sec:orgc5a50c0}
\subsection{Peterson Lock in more selfish}
\label{sec:org7ef2451}
\subsubsection{Does the Peterson Lock still guarantee mutual exclusion?}
\label{sec:org40f398a}

Answer: No, because if each thread can set its own turn to self,
there is a possibility of an overlap of the critical sections, if
one thread happens its to set its turn to self after the first
thread has entered but not left the critical section.

\begin{center}
\includegraphics[width=.9\linewidth]{/mnt/c/Users/sgraetz/Desktop/blatt3_advp_i.png}
\end{center}
\subsubsection{Is the lock still starvation free?}
\label{sec:orgc0a36a8}
Answer: Yes, because either the mutual exclusion is not guaranteed
but both threads can execute to completion - though be it with
incorrect results. Or The release of the first thread will free
the second from its limbo. 

\begin{center}
\includegraphics[width=.9\linewidth]{/mnt/c/Users/sgraetz/Desktop/blatt3_advp_ii.png}
\end{center}
\newpage
\subsection{Atomic Peterson Lock}
\label{sec:orgde4c298}
\subsubsection{Implementation}
\label{sec:orgff981ee}
Implement the lock using C++ atomics. Note that the pseudocode
representation assumes a sequentially consistent memory
model. Devise an approach for testing the correctness of your
implementation. 
\begin{verbatim}
#include <atomic>
#include <iomanip>
#include <iostream>
#include <thread>
#include <vector>

class AtomicPetersonLock {
private:
    std::atomic<int> turn;
    std::atomic<bool> interested[2];

public:
    AtomicPetersonLock() {
        turn.store(0, std::memory_order_release);
        interested[0].store(false, std::memory_order_release);
        interested[1].store(false, std::memory_order_release);
    }

    ~AtomicPetersonLock() {}

    void aquire(int thread_id) {
        int other = 1 - thread_id;
        interested[thread_id].store(true, std::memory_order_acquire);
        turn = other;

        int spins = 0;
        while (interested[other].load(std::memory_order_acquire) &&
            turn.load(std::memory_order_acquire) == other) {
            if (spins++ % 10000000 == 0) {
                std::cout << "thread" << thread_id << " spined for " << spins << "spins\n";
            }
        }
        std::cout << "aqurired by thread" << thread_id << "\n";
    }

    void release(int thread_id) {
        std::cout << "released by thread" << thread_id << "\n";
        interested[thread_id].store(false, std::memory_order_release);
    }
};
\end{verbatim}
\newpage
\subsubsection{Test Implementation}
\label{sec:orgbc0371d}
Compare the 100000000 add 1 operations by each thread to a global
variable. If no thread is used the  result is different in each
iteration. If the atomic Peterson lock implementation is used the
result is correctly 200000000 each time – so it is not just
coincidental. Also each iteration run without starvation and
terminates.
\begin{verbatim}
void test_atomic_peterson_lock(void) {
    int global_int = 0;
    AtomicPetersonLock lock = AtomicPetersonLock();

    std::cout << "\n\nAtomic PetersonLock: \n";
    auto func = [&global_int, &lock](int thread_id) {
        lock.aquire(thread_id);
        for (auto i = 0; i < 100000000; ++i) {
            global_int += 1;
        }
        lock.release(thread_id);
    };
    auto func_without_lock = [&global_int](int thread_id) {
        for (auto i = 0; i < 100000000; ++i) {
            global_int += 1;
        }
    };

    std::cout << "Compare with lock vs without vs should be:\n";
    std::cout << "| Atomic Peterson Lock |  No Lock              | Expected "
        "value        |\n";
    for (auto i = 0; i < 10; ++i) {
        global_int = 0;
        {
            std::thread t1;
            std::thread t2;
            t1 = std::thread{func, 0};
            t2 = std::thread{func, 1};
            t1.join();
            t2.join();
        }
        int result = global_int;
        global_int = 0;
        {
            std::thread t1;
            std::thread t2;
            t1 = std::thread{func_without_lock, 0};
            t2 = std::thread{func_without_lock, 1};
            t1.join();
            t2.join();
        }
        std::cout << "| " << result << std::setw(14) << " | ";
        std::cout << global_int << std::setw(15) << " | ";
        std::cout << 200000000 << std::setw(15) << " |\n";
    }
}

auto main(int argc, char *argv[]) -> int {
    test_atomic_peterson_lock();
    return 0;
}

\end{verbatim}
\section{MCS Lock}
\label{sec:org3bc535f}
\subsection{Implement the [MCS] lock using C++ atomics.}
\label{sec:orgf8f5c23}
\begin{verbatim}
#include <atomic>
#include <iostream>
#include <thread>

struct QNode { std::atomic<bool> wait; std::atomic<QNode *> next; };

class AtomicMCSLock {
private:
    std::atomic<QNode *> tail;

    QNode *swap(std::atomic<QNode *> &_tail, QNode *_p) {
        return _tail.exchange(_p, std::memory_order_acq_rel);
    }

    bool cas(std::atomic<QNode *> &_tail, QNode *_expected, QNode *_desired) {
        return _tail.compare_exchange_weak(_expected, _desired,
            std::memory_order_acq_rel);
    }

public:
    void acquire(QNode *p) {
        p->next.store(nullptr);
        p->wait.store(true);

        QNode *prev = swap(tail, p);

        if (prev) {
            prev->next.store(p, std::memory_order_release);
            while (p->wait.load(std::memory_order_acquire)) {
                //
            }
        }
        std::cout << "acquired\n";
    }

    void release(QNode *p) {
        QNode *succ = p->next.load(std::memory_order_acquire);

        if (!succ) {
            auto desired = p;
            if (cas(tail, desired, nullptr)) {
                return;
            }
            do {
                succ = p->next.load(std::memory_order_acquire);
            } while (succ == nullptr);
        }
        succ->wait.store(false, std::memory_order_release);
        std::cout << "released\n";
    }
};
\end{verbatim}
\subsection{Test the implementation}
\label{sec:orga549f97}
Here I ran out of time\dots{}
\begin{verbatim}
auto main() -> int {
    std::thread t1;
    std::thread t2;

    AtomicMCSLock mcs = AtomicMCSLock();
    QNode node;
    int global = 0;

    auto func = [&mcs, &global, &node]() {
        mcs.acquire(&node);
        for (auto i = 0; i < 100000000; ++i) {
            global += 1;
        }
        mcs.release(&node);
    };

    std::cout << "global = " << global << "\n";
    t1 = std::thread{func};
    t2 = std::thread{func};
    t1.join();
    t2.join();
    std::cout << "global = " << global << "\n";

    return 0;
}
\end{verbatim}
\end{document}
